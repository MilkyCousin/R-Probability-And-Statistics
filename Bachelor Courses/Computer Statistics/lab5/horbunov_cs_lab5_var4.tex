\documentclass[12pt]{article}

\usepackage[utf8]{inputenc}
\usepackage[T1]{fontenc}
\usepackage[english,ukrainian]{babel}
\usepackage[left=2cm,right=2cm,top=1cm,bottom=2cm,bindingoffset=0cm]{geometry}

\usepackage[rightcaption]{sidecap}

\usepackage{amsmath}
\usepackage{amssymb}
\usepackage{bigints}
\usepackage{cancel}
\usepackage{dsfont}
\usepackage{graphicx}
\usepackage{hyperref}
\usepackage{pifont}

\graphicspath{ {./report/pictures/} } 

\newcommand{\mprb}[1]{\mathbb{M}\left[#1\right]}
\newcommand{\vprb}[1]{\mathbb{D}\left[#1\right]}
\newcommand{\prb}[1]{\mathbb{P}\left(#1\right)}
\newcommand{\cov}{\text{cov}}
\newcommand{\sign}{\text{sign}}
\newcommand{\tran}{^{\top\kern-\scriptspace}}
\newcommand{\mtran}{^{-\top\kern-\scriptspace}}
\newcommand{\tdef}[1]{Означення #1}
\newcommand{\twrn}[1]{Зауваження #1}
\newcommand{\tlem}[1]{Лема #1}
\newcommand{\tstm}[1]{Твердження #1}
\newcommand{\tthe}[1]{Теорема #1}
\newcommand{\sprd}[2]{\left\langle #1, #2\right\rangle}
\newcommand{\bnrm}[1]{||#1||}

\newcommand{\ueq}[1]{\mathrel{\stackrel{\makebox[0pt]{\mbox{\normalfont\tiny #1}}}{=}}}

\title{Лабораторна робота №5\\ з дисципліни ''комп'ютерна статистика''\\Варіант №4}
\author{
        Горбунова Даніела Денисовича\\
        4 курс бакалаврату\\
        група "комп'ютерна статистика"
}
\date{\today}

\begin{document}
\maketitle

\section{Вступ.}

У даній роботі побудовано критерій відношення вірогідностей для перевірки простих гіпотез про значення ймовірності успіху в біноміальному розподілі. Визначено обсяг вибірки, при якому тест матиме ймовірності похибок першого та другого роду, які не перевищують 0,05.

\section{Побудова критерію.}

Нехай $X = (\xi_1, \ldots, \xi_n)$ - кратна вибірка. Обчислимо функцію вірогідності за біноміальним розподілом $Bin(m, \theta), \> \theta \in (0, 1)$:
\begin{equation*}
L(X, \theta) = \prod_{j = 1}^n C_m^{\xi_j} \theta^{\xi_j} (1 - \theta)^{m - \xi_j} = \left(\prod_{j = 1}^n C_m^{\xi_j}\right) \theta^{\sum_{j = 1}^n \xi_j} (1 - \theta)^{nm - \sum_{j = 1}^n \xi_j} 
\end{equation*}
Беремо $p, q \in (0,1): p < q$. Обчислимо за ними логарифмічну функцію вірогідності:
\begin{align*}
LR(X) &= \dfrac{L(X, q)}{L(X, p)} = \dfrac{\cancel{\left(\prod_{j = 1}^n C_m^{\xi_j}\right)} q^{\sum_{j = 1}^n \xi_j} (1 - q)^{nm - \sum_{j = 1}^n \xi_j} }{\cancel{\left(\prod_{j = 1}^n C_m^{\xi_j}\right)} p^{\sum_{j = 1}^n \xi_j} (1 - p)^{nm - \sum_{j = 1}^n \xi_j} } = \left(\dfrac{q(1 - p)}{p(1 - q)}\right)^{\sum_{j = 1}^n \xi_j}\left(\dfrac{1 - q}{1 - p}\right)^{nm} \Rightarrow
\\
\ln{LR(X)} &= \ln{\dfrac{L(X, q)}{L(X, p)}} = \sum_{j = 1}^n \xi_j \ln{\left(\dfrac{q(1 - p)}{p(1 - q)}\right)} + mn\ln{\left(\dfrac{1 - q}{1 - p}\right)}
\end{align*}
Зауважимо, що $\sum_{j = 1}^n \xi_j$ є монотонно зростаючою функцією відносно $\ln{LR(X)}$, тому критерій відношення вірогідностей запишемо у вигляді:
\begin{equation*}
\pi_C(X) = \mathds{1}\left\lbrace\sum_{j = 1}^n \xi_j > C_{\alpha}\right\rbrace
\end{equation*}
Тобто якщо сума спостережень перевищує заданий поріг рівня $\alpha$, тоді приймається нульова гіпотеза ($\pi_C(X) = 0$), інакше приймається альтернатива.
\newpage
Зафікуємо рівень значущості $\alpha \in (0, 1)$. Тоді поріг рівня $\alpha$ визначимо з умови:
\begin{equation*}
\alpha = \alpha_{\pi} = \mathbb{M}_p\pi_C(X) = \mathbb{P}_p\left\lbrace \sum_{j = 1}^n \xi_j > C_{\alpha} \right\rbrace
\end{equation*}
Ймовірність похибки другого роду для цього тесту:
\begin{equation*}
\beta_{\pi} = 1 - \mathbb{M}_q\pi_C(X) = \mathbb{P}_q\left\lbrace \sum_{j = 1}^n \xi_j \leqslant C_{\alpha} \right\rbrace
\end{equation*}
Якщо $\{\xi_j\}_{j=1}^n$ - н.о.р. з біноміальним розподілом $Bin(m, \theta)$, тоді $\sum_{j = 1}^n \xi_j \sim Bin(nm, \theta)$. Це нам може спростити життя хоча б в тому сенсі, що розподіл статистики відомий у явному вигляді. Застосуємо наведені раніше міркування для програмної реалізації тесту:
\begin{verbatim}
# H0 : Binom(0.5, 2), p := 0.5
# H1 : Binom(0.6, 2), q := 0.6
# n = 65
set.seed(0)
log.lr.test <- function(x, m, p, q, alpha = 0.05, to.print = F)
{
  x.sum <- sum(x)
  n <- length(x)
  c.alpha <- qbinom(1 - alpha, n * m, p)
  h.res <- c.alpha >= x.sum
  if(to.print)
  {
    print(paste("sum(X) = ", x.sum, 
                ifelse(h.res, "<=", ">"), 
                c.alpha, "= c_{alpha}")) 
  }
  list(hypothesis = 1 - h.res, statistic = x.sum) # 0 - H0, 1 - H1
}
p <- 0.5
q <- 0.6
m <- 2
n.fixed <- 65
# Приклад застосування
print(log.lr.test(rbinom(65, 2, 0.5), 2, 0.5, 0.6, to.print = T))
# [1] "sum(X) =  68 <= 74 = c_{alpha}"
# $hypothesis
# [1] 0
# $statistic
# [1] 68
\end{verbatim}
\newpage
\begin{figure}[h]
\centering
\includegraphics[scale=0.65]{lrhistempirical}
\caption{Гістограма $LR$-статистик в залежності від обраної гіпотези. Зліва - гістограма за нульовою гіпотезою, справа - гістограма за розподілом альтернативи. Зелена вертикальна лінія - поріг тесту.}
\end{figure}

За допомогою імітаційного моделювання, після проведення $B = 10^4$ експериментів з використанням тесту відношення вірогідностей, отримані наступні оцінки імовірностей першого та другого роду:

\begin{verbatim}
# Імітаційне моделювання
B <- 10000
# Генеруємо вибірки за розподілом при виконанні нульової (альтернативної) гіпотези
# У відповідні масиви вносимо результат тесту та значення статистики
counts.0.1 <- numeric(B)
x.stat.0 <- numeric(B)
for(j in 1:B)
{
  x.0 <- rbinom(n.fixed, m, p); r <- log.lr.test(x.0, m, p, q)
  counts.0.1[j] <- r$hypothesis; x.stat.0[j] <- r$statistic
}
counts.1.1 <- numeric(B)
x.stat.1 <- numeric(B)
for(j in 1:B)
{
  x.1 <- rbinom(n.fixed, m, q); r <- log.lr.test(x.1, m, p, q)
  counts.1.1[j] <- r$hypothesis; x.stat.1[j] <- r$statistic
}

# Оцінка для ймовірності помилки першого роду
print(
  paste("Estimated probability of I type error:", 
        mean(counts.0.1))
  )
# "Estimated probability of I type error: 0.046"

# Оцінка для ймовірності помилки другого роду
print(
  paste("Estimated probability of II type error:", 
        1 - mean(counts.1.1))
  )
# "Estimated probability of II type error: 0.2646"

# Гістограма статистик в залежності від параметра ймовірності
min.stat <- min(x.stat.0, x.stat.1)
max.stat <- max(x.stat.0, x.stat.1)

hist(x.stat.0, col = 'red', xlim = c(min.stat, max.stat), ylim = c(0, 0.08),
     freq = F, breaks = 20, main = "Histogram of LR-statistics", angle = -45,
     density = 15, xlab = "Value", panel.first = grid())
hist(x.stat.1, col = 'blue', xlim = c(min.stat, max.stat), ylim = c(0, 0.08),
     freq = F, breaks = 20, add = T, density = 15, angle = 45)

# Емпірична оцінка порогу тесту
c.b <- quantile(x.stat.0, 1 - 0.05)
print(c.b)
# 95% 
# 74 
abline(v = c.b, lwd = 2, col = 'darkgreen')
\end{verbatim}

Практика непогано узгоджується з теорією в сенсі отриманих результатів, близьких до очікуваних. 

\end{document}